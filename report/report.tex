%%
%%  Example paper
%%
%%

%%%%%%%%%%%%%%%%%% Usenix style %%%%%%%%%%%%%%%%%%%%%%%%%%%%%%%%%
\documentclass[10pt,twocolumn,a4paper]{article}
\usepackage{styles/usenix-style}

\author{Jannik Wibker}

%%%%%%%%%%%%%%%%%% Document %%%%%%%%%%%%%%%%%%%%%%%%%%%%%%%%%%%%%%%%%%%
% TODO: Change draft to final before submitting final version.
\usepackage[draft]{styles/ka-style}
\usepackage{cite,xspace,ifthen,graphicx,listings}

\usepackage[
   pdfauthor={Jannik Wibker},
   pdftitle={Kernel Address Space Layout Randomization: an Overview},
   pdfsubject={KASLR},
   pdfkeywords={Linux, Kernel, ASLR, KASLR, DrK, KPTI}
]{hyperref}

\begin{document}

\title{ Kernel Address Space Layout Randomization: an Overview }

\newcommand{\todo}[1]{{\texttt{[#1]}}}
\newcommand{\code}[1]{{\tt \small{#1}}}

\maketitle
%\draftfooter

\begin{abstract}
KASLR and ASLR are both important measures to strengthen the security of operating systems.
They are however not without flaws and can be circumvented.
In this report I will give an overview of the state of KASLR, its origins and differences in implementation across different operating systems, detail an attack against KASLR called DrK and touch on KPTI (previously KAISER), which is a later introduced mitigation against said attack and multiple others such as Meltdown.
DrK abuses the fact that TSX (Intel Transactional Synchronization Extension) does not notify the kernel of page faults and access violations but rather solely the exception handler of the transaction in user space. Using this DrK can do a timing attack and scan the address space for allocated pages and obtain whether a page is executable or not if mapped.
From this the addresses of the kernel as well as its modules can be inferred.
KPTI mitigates this by reducing the footprint of the kernel in user space and thus reducing the amount of information that can be obtained by DrK, Meltdown and similar attacks.

% add references for: DrK, Meltdown, KAISER, (maybe TSX)

\end{abstract}

\section{Introduction}\label{sec:introduction}
\section{Background}\label{sec:background}

\begin{figure}[htbp]
  \centering
  \fbox{\parbox{.8\columnwidth}{
      Here you can include a sample figure.  Use something like
      \begin{center}
        \code{$\backslash$includegraphics[scale=.8]\{template\}}
      \end{center}
      to include an encapsulated postscript figure.  The \emph{scale}
      argument can be used for scaling the picture, although it
      may scale the font incorrectly.
    }}
  \caption{Sample Figure}
  \label{fig:sample}
\end{figure}


\lstset{language=C, basicstyle=\ttfamily,
        string=[b]', showspaces=false, showtabs=false,
        caption={A sample code snippet}, captionpos=b}
\begin{lstlisting}
/* code snippet  */
while (!sleep)
	sleep++;
\end{lstlisting}

\begin{figure}[hbt]
\centering
\caption{Sample figure automatically from Windows prn.\label{plot:fig}}
\end{figure}

\section{Related Work}\label{sec:relwork}

Works \cite{xen03virtualization} and \cite{pratt2005xaa} are relevant but
different.

\section{Approach}

\section{Conclusion}\label{sec:conclusion}

\bibliographystyle{abbrv}
\bibliography{report}
%\footnotesize
\end{document}
