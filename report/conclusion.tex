Both KASLR and ASLR provide very valuable security against attacks.
The possibility of derandomization attacks existing doesn't negate the positive impact ASLR and KASLR pose outright, it just slightly lessens their effect.
As with many other vulneratibilities a chain of them has to exist with each one compromising a different layer or part of the system.
Thus a derandomization attack is only a single piece of the puzzle and on its own has no direct impact, only when paired with other exploits making use of the gained information.
It is thus not the end of the world if a derandomization is found and not immediately fixed or mitigated as there are more things that have to go wrong before an attack is possible.
And even with all of the pieces in place having more things to bypass means more complexity on the attacker side which in turn means greater chance of failure or detection as is the case with other software as well.
