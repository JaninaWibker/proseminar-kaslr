The DrK is an attack against KASLR which derandomizes the kernel address space.
It has been proposed in the paper ... by ...
Most of the code necessary to perform the attack can be found on GitHub (TODO link).

The attack builds upon Intel TSX (TODO reference) and performs a timing attack to infer the state of a page (mapped vs. unmapped and executable vs. non-executable).
This works because with TSX the kernel does not get notified of Page Faults and Access Violations.
Furthermore the time it takes the MMU (Memory Managment Unit) to throw a Page Fault exception slightly differs from the time it takes to throw an Access Violation exception.
Using this the status of a page can be inferred.

\subsection{Intel TSX}

Intel TSX stands for the Intel Transaction Synchronization Extension(s).
It is an interface using which code can be run transactionally, i.e. segmented into clearly defined transactions, aborted and rolled back including all memory writes if so needed.

Aborting and rolling back happens either manually using a directive to abort the transaction or automatically in the case of an exception such as a Page Fault.

\lstset{language=C, basicstyle=\ttfamily,
        string=[b]', showspaces=false, showtabs=false,
        caption={Example code for Intel TSX}, captionpos=b}
\begin{lstlisting}
// start transaction, if rolled back
// execution continues from here
// but with a different return value
// (status code of transaction)
if (_xbegin() == _XBEGIN_STARTED) {

  ... // do work

  // commit transaction
  _xend();

  // or abort with status code
  // (=> rollback)
  _xabort(status)
} else {
  // abort handler
}
\end{lstlisting}

The exception being directly handled by an abort handler in userspace is far from the norm, otherwise this will always notify the kernel which can then decide to terminate the process or take some other kind of action.
With TSX this is not possible as the kernel does not ever get notified of this exception occuring.
This can pose great potential for abuse if used in a nefarious manner as this behavior differs from the status quo and assumptions used elsewhere might not hold with this.
Mechanisms which otherwise rely on the kernel getting notified about page faults such as Demand Paging will not work correctly.

\subsection{Timing differences}

The abort handler does not get information about what exception occured.
Page Faults point to the page being unmapped, while Access Violations point to the page being mapped but inaccessible from userspace.
Differentiating a Page Fault from an Access Violation will have to be done in some other way.
The time the MMU needs to generate a Page Fault for a given memory access differs from the time it takes to generate an Access Violation as the prior one can be deduced faster.
This difference can be measured and used to tell both apart.

Checking for execution status of the page works similarly.
For a page that is known to be mapped (but inaccessible from userspace, meaning it belongs to the kernel address space) trying to jump to an address contained in the page will take a different amount of time until the exception is generated based on whether the page is executable or not.

\subsection{Crafting the attack}

TODO: code blocks

\subsection{Interpreting the results}

TODO: figure with timing data points
TODO: talk about thresholds to determine category
