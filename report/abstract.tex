KASLR (Kernel Address Space Layout Randomization) and ASLR (Address Space Layout Randomization) are both important measures to strengthen the security of operating systems.
They are however not without flaws and can be circumvented.
In this report I will give an overview of the state of KASLR, its origins and differences in implementation across different operating systems, detail an attack against KASLR called DrK introduced by Jang et al. \cite{drk} and touch on KPTI (previously KAISER by Gruss et al. \cite{kaiser}), which is a later introduced mitigation for said attack and multiple others such as Meltdown \cite{meltdown}.
